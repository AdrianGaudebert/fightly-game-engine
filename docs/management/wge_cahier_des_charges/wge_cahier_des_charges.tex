\documentclass[a4paper,10pt]{report}
\usepackage[utf8x]{inputenc}
\usepackage[francais]{babel}
\usepackage[T1]{fontenc}

% Title Page
\title{Web Game Engine - Cahier des charges}
\author{Ilyas Boutebal \and Maxime Colin \and Adrian Gaudebert \and Youness Hamri \and Van-Duc Nguyen}


\begin{document}
\maketitle

\tableofcontents

\chapter{Contexte}

\section{Le Jeu vidéo}
Le marché du jeu vidéo s’est beaucoup développé depuis l’arrivée de l’ancêtre Pong ou du célèbre Tetris. On joue aujourd’hui à des jeux de plus en plus beaux, demandant de plus en plus de ressources à nos machines. Mais on voit aussi d’autres utilisations du jeu vidéo arriver : le Serious Gaming, par exemple, est une branche en pleine expansion en ce moment. Le Social Gaming également, représenté activement par les nombreux jeux basés sur Facebook et ses possibilités en terme de diffusion.

Si on s’essayait à une catégorisation rapide des jeux vidéo, voici ce qui pourrait ressortir : les jeux consoles, les jeux PC “classiques”, les jeux en ligne (MMO en tous genres compris). Dans cette dernière catégorie, on trouve beaucoup de types de jeux différents : les MMO dans le genre de World of Warcraft, les jeux multijoueur comme Counter Strike ou Team Fortress, les modes multi des jeux principalement solo, et les jeux par navigateur.

\section{Les jeux par navigateur}
Le jeu par navigateur, ou Browser Game en anglais, se joue par définition dans un navigateur Internet. On retrouve donc dans cette catégorie les jeux “Facebook” cités plus haut, la majorité des jeux en Flash, mais également un très grand nombre de jeux que nous appellerons les Jeux Web, puisqu’ils se basent sur les technologies du Web ouvert. Quelques exemples de jeux web relativement connus : Ogame, Travian, ou le français Hordes.

Ces trois exemples ne sont cependant pas représentatifs de la diversité que l’on trouve dans les jeux web. Les jeux d’élevage virtuel, par exemple, sont extrêmement nombreux sur le Web. À quoi est due cette profusion de jeux web ? Majoritairement à la simplicité d’accès du développement web. Faire un jeu web, en soit, c’est développer un site web. Or, avec des technologies très répandues comme PHP, HTML et CSS, avec toutes les ressources que l’on trouve autour de ces dernières (il suffit de regarder les tutoriels du Site du Zéro pour s’en rendre compte), il est très simple pour une personne un peu motivée de créer son propre jeu web.

Malheureusement, s’il est simple de créer un jeu web “basique”, les développeurs sont rapidement limités par les technologies qu’ils utilisent. Il est quasiment impossible de faire du temps réel avec PHP, le couple HTML / CSS n’est pas adapté à l’affichage d’effets spéciaux ou d’animations complexes, et si l’arrivée récente de frameworks JavaScript comme jQuery a permis de repousser un peu ces limites techniques, cela ne résout pas le problème.

\section{Les technologies du web}
Il existe une solution simple aux limitations techniques actuelles des technologies web : se tourner vers le futur et utiliser de nouvelles technologies, pas encore éprouvées, mais qui permettent d’aller beaucoup plus loin dans la création de nos jeux web.

HTML 5, la nouvelle mouture du langage de description des pages web, apporte au développeur un très grand nombre de fonctionnalités clés : le temps réel dans le navigateur avec les WebSockets, les manipulations graphiques avancées avec le SVG et les Canvas, la vidéo et le son avec Video et Audio, et bien d’autres.

La version 3 du langage CSS offre elle aussi d’alléchantes nouveautés : les animations, les transitions, les polices particulières, ainsi que tous les effets graphiques amplement simplifiés.

\chapter{Le projet}

\section{Objectifs}
Toutes les avancées actuelles du web ouvrent indéniablement les portes à des jeux de plus en plus impressionnants, de plus en plus profonds, le tout se passant dans un navigateur ! Il faudra cependant du temps pour que les développeurs maîtrisent tout ceci, pour que des outils facilitant l’utilisation de ces technologies apparaissent, et donc pour que l’accès à toutes ces possibilités pour nos jeux devienne simple.

Voici donc où se place ce projet : notre objectif est de fournir un outil complet, permettant de créer de façon simple et rapide un jeu profitant des dernières avancées en matière de technologie web. Il constitura un ensemble d'outils qui fournissent un cadre de développement et des fonctionnalités permettant d'implémenter tout ou partie d'un jeu.

En accord avec cet objectif d'orienté web, il sera important de respecter les standards du web ainsi que de proposer une large compatibilité avec les principaux navigateurs du marché.

Le moteur sera exclusivement conçu pour des jeux de stratégie au tour par tour.

De part sont cadre de développement, le projet sera sous licence libre.

Afin de proposer un outils évolutif et ouvert, nous mettrons l'accens sur ses possibilités d'extension.


\section{Spécifications}

\subsection{Fightly}
Le projet dans sa version initiale consistait à développer
un jeu au tour par tour, par navigateur sur internet appelé Fightly, mais plusieurs 
raisons nous ont amener à ne pas prendre en compte des parties du jeu
que nous avons estimé sortant du cadre du Master,
notamment les aspects graphiques, vidéo, audio, forum de discutions et 
l'initialisation des parties. Le projet est donc réduit à l'essentiel, 
c'est à dire fournir un moteur de jeu qui permet d'implémenter les 
fonctionnalités de tout ce qui est en lien avec le déroulement du jeu. 
Autrement dit, on se limitera à la mécanique du jeu.\smallskip \smallskip

Fightly est un projet de jeu qui se déroule sur une carte choisie 
au hasard parmi un ensemble de cartes précrées. Les cartes sont 
composées de cases hexagonales contigues où chacune possède un type de terrain
particulier ralentissant ainsi ou intérdisant les déplacement sur certains 
types de terrains. 
Les joueurs choisisent leurs unités de combats composées de fantassins, 
archers et chevaliers ayant chacuns des caractéristiques et des capacités différentes.
ils les placent sur la carte, et s'affrontent chacun son tour en les faisant 
déplacer et attaquer suivant des règles de portée bien précises.

Pour plus de détails sur Fightly, voici le lien disponible sur le forum du jeu.

\subsection{Serveur}

Le serveur sera la partie du moteur qui fera tourner les parties et permettra la communication et les interraction entre les joueurs. Il aura pour rôle de contrôler les parties, lier les clients et gérer leur actions. Il aura également le rôle d'arbitre dans une partie. 

Le serveur sera séparer en deux grandes parties : Une partie Noyau et une partie Monde.

>Diagramme6.png<

Package Noyau

La partie Noyau du serveur gère les fonctionnalités de base du moteur. Elle contient un ensemble de modules qui sont a priori indépendant du type de jeu implémenté. De plus, ce noyau doit permettre les interactions entre les règles et les modules de la partie Monde, quels que soient ces derniers. 

La sous-partie communication du serveur doit permettre d'envoyer et de recevoir des messages. Cette partie devra également accepter et gérer les connexions clients.

La sous-partie protocole sera charger d'adapter les messages. C'est à dire encapsuler nos message dans le protocole choisi pour l'envoie de message et extraire le contenu d'un message reçu. Il fera le lien entre les messages sur le réseaux et les messages comprehensibles par le client et le serveur.

La sous-partie règles sera charger de stocker et traiter les règles du jeu. Elle contiendra des fonctionnalités permetant de créer une base de règles. Ces règle pourront être ajouté et supprimé individuellement ou charger à partir d'un fichier de règle. La partie règle devra également valider (ou invalider) les requêtes du client.



Package Monde

La partie Monde est totalement dynamique : les modules qu'elle contient sont fonction du jeu à développer, et peuvent donc changer du tout au tout. Les modules de cette partie doivent donc pouvoir être appelés depuis le noyau sans avoir à faire autre chose que de la configuration. 

Les modules présents dans cette partie implémentent des fonctionnalités du jeu. Ceux choisis ici sont donc ceux qui serviront, a priori, au jeu Fightly.

Cette partie aura des fonctionnalités permettant de gérer une partie. Ces fonctionnalité concerneront le chargement et la sauvegarde d'une partie. Elle gera également la connexion et la déconnexion à une partie.

Une sous-partie sera dédié à la gestion de la carte. Elle ne disposera que d'une seule fonctionnalité permettant de charger la carte à partir d'un fichier.

Le serveur disposera de fonctionnalités pour gérer les unités des joueurs. Cette sous-partie permettra de créer, supprimer ou déplacer les unités.

Le serveur devra également gérer les joueurs. Une sous-partie permmettra de créer et supprimer des joueurs dans une partie. Elle devra également gerer l'abandon d'un joueur durant une partie. C'est à dire retirer le joueur et ses unités de la partie.



\subsection{Client}

A compléter par Ilyas

\end{document}          
