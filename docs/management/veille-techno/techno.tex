\documentclass[a4paper,10pt]{report}
\usepackage[utf8x]{inputenc}
\usepackage[francais]{babel}
\usepackage{fullpage}
\usepackage[none]{hyphenat}
\usepackage{listings}
\usepackage{url}

% Title Page
\title{Communication}
\author{Maxime COLIN}


\begin{document}
\maketitle

\begin{abstract}
Ce document présente les différentes technologies envisagées pour notre projet.
\end{abstract}


\chapter{Compatibilité}


\chapter{Client}


\chapter{Serveur}


\chapter{Communication}

Cette partie concerne les technologies de communication client/serveur et 
client/client. Les différentes pistes retenu sont les WebSockets, Ahax, 
Opera Unite.

  \section{WebSocket}

    \subsection{Présentation}

WebSocket est une technologie fournissant un canal de communication bidirectionnel 
et fullduplex à travers un socket TCP. Il a été conçu pour être implémenté dans 
les navigateurs et serveurs web, mais il peut être utilisé n'importe quelle 
application client ou serveur. L'API WebSocket est en phase de standardisation 
par le W3C et le protocole par IETF.


Un tel canal de communication permet :
\begin{itemize}
  \item la notification au client d'une modification d'état du serveur
  \item l'envoie de données du serveur au client sans que celui ci n'ai à faire de requête
\end{itemize}


    \subsection{Implémentation et support coté client}

Le protocoles WebSocket est implémenté dans les navigateurs Chrome 4, Safari 5, 
Firefox 4 et Opera 11. Son support est néanmoins désactivé dans Firefox 4 et 
Opera 11 pour de raison de sécurité. Il est possible de l'activé via dans les 
paramètres des deux navigateurs. Internet Explorer ne supporte pas WebSocket.


    \subsection{Implémentation et support coté serveur}

Le protocole nécessite également d'être implémenté côté serveur pour être 
utilisé. Il existe plusieurs implémentations côté serveur de WebSocket, 
dans différents langages (Java, Python, PHP, Javascript, ...) et sous 
différentes formes (extension apache, serveur entier, script, ...).


Quelques implémentations :
\begin{itemize}
  \item GNU WebSocket4J, une implémentation du protocole WebSocket en Java. 
  \item pywebsocket3, une implémentation en Python sous la forme d'une extension pour le serveur  Apache.
  \item jWebSocket, implémentation Java côté serveur et JavaScript/HTML5 côté client.
  \item Implémentation de WebSocket avec node.js
\end{itemize}


    \subsection{Sécurité}

WebSocket comporte à l'heure actuel une faille de sécurité dans la phase 
de ``handshacke'' permettant de remplacer un fichier javascript par un malware. 
La faille se situe au niveau de l'API elle même. C'est pourquoi son support 
est désactivé par défaut Firefox 4 et Opera 11 jusqu'à ce que la faille 
soit comblé.


  \section{Ajax}

    \subsection{Présentation}

Ajax (Asynchronous Javascript and XML) est un rassemblement de different outils 
et méthode de conception permettent de construire des application web dynamique 
basé sur différentes technologies web côté client.

Ajax est une combinaison de technologie telle que JavaScript, CSS, XML, DOM et 
XMLHttpRequest dans le but de réaliser des applications Web qui offrent une 
maniabilité et un confort d'utilisation supérieur à ce qui se faisait jusqu'alors.

DOM et JavaScript sont utilisés pour modifier l'information présentée dans le navigateur 
par programmation. L'objet XMLHttpRequest est utilisé pour dialoguer de manière 
asynchrone avec le serveur Web. La notation XML est utilisée pour structurer les 
informations transmises entre le serveur Web et le navigateur.

En alternative au format XML, les applications Ajax peuvent utiliser les fichiers texte ou JSON.

  \subsection{Support}

Les applications Ajax fonctionnent sur tous les navigateurs Web qui mettent en oeuvre les 
technologies décrites précédemment, parmi lesquels Mozilla Firefox, Internet Explorer, 
Konqueror, Google Chrome, Safari et Opera.


  \section{Opera Unite}

    \subsection{Présentation}

Opera Unite est une technologie qui transforme un navigateur en serveur 
Web personnel. Avec Opera Unite, on devient à la fois client et serveur, 
à la fois visiteur et hôte. On reçoit du contenu du Web, et on en fournit 
également. On garde le contrôle : les données restent sur votre ordinateur, 
et on décide avec qui on désire les partager. 

    \subsection{Implémentations}

Opera Unite est implémenté dans le navigateur Opera depuis la version 10.


    \subsection{Conclusion sur Opera Unite}

Opera Unite est au final un service de partage de contenu et non de 
communication client/serveur, cette solution est donc inadaptée à nos 
besoins. De plus, cette fonctionnalité n'est disponible que sur le 
navigateur Opera.

  \section{Conclusion}

En conclusion, la technologie WebSocket semble la plus intéressante pour notre projet.
Elle semble adapté à nos besoins. Néanmoins le fait que le protocoles soit 
encore en phase de développement et le fait qu'il soit désactivé par défaut 
sur certains navigateurs pourrait poser problème.

Ajax semble également correspondre à certaines de nos attentes, mais dispose de plus
de restriction au niveau communication client/serveur.

\chapter{Formats de données}

  \section{JSON}

    \subsection{Présentation}

JSON (JavaScript Object Notation) est un format de données textuel, 
générique, dérivé de la notation des objets du langage ECMAScript. 
Il permet de représenter de l’information structurée. Créé par Douglas 
Crockford, il est décrit par la RFC 4627 de l’IETF.


Un document JSON ne comprend que deux éléments structurels : 
\begin{itemize}
 \item des ensembles de paires nom / valeur ; 
 \item des listes ordonnées de valeurs. 
\end{itemize}


Ces mêmes éléments représentent 3 types de données : 
\begin{itemize}
  \item des objets ; 
  \item des tableaux ; 
  \item des valeurs génériques de type tableau, objet, booléen, nombre, chaîne ou null. 
\end{itemize}

Le format JSON est très facilement exploitable et manipulable en Javascript. 
Un document JSON représente un objet. Il est donc potentiellement plus facile 
à manipuler qu'un document XML.

    \subsection{Exemple}

\begin{verbatim}
{"menu": {
   "id": "file",
   "value": "File",
   "popup": {
     "menuitem": [
       {"value": "New", "onclick": "CreateNewDoc()"},
       {"value": "Open", "onclick": "OpenDoc()"},
       {"value": "Close", "onclick": "CloseDoc()"}
     ]
   }
 }}
\end{verbatim}

    \subsection{Sources}

\begin{itemize}
 \item \url{http://fr.wikipedia.org/wiki/Json}
\end{itemize}


  \section{XML}

    \subsection{Présentation}

Extensible Markup Language est un langage informatique de balisage générique. 
Il sert essentiellement à stocker/transférer des données de type texte Unicode 
structurées en champs arborescents. Ce langage est qualifié d’extensible car 
il permet à l'utilisateur de définir les balises des éléments. L'utilisateur 
peut multiplier les espaces de nommage des balises et emprunter les définitions 
d'autres utilisateurs

    \subsection{Exemple}

\begin{verbatim}
<menu id="file" value="File">
  <popup>
    <menuitem value="New" onclick="CreateNewDoc()" />
    <menuitem value="Open" onclick="OpenDoc()" />
    <menuitem value="Close" onclick="CloseDoc()" />
  </popup>
</menu>
\end{verbatim} 

    \subsection{Sources}

\begin{itemize}
 \item \url{http://fr.wikipedia.org/wiki/Xml}
\end{itemize}


  \section{HTML}

    \subsection{Présentation}

L’Hypertext Markup Language, généralement abrégé HTML, est le format de données 
conçu pour représenter les pages web. C’est un langage de balisage qui permet 
d’écrire de l’hypertexte, d’où son nom. HTML permet également de structurer 
sémantiquement et de mettre en forme le contenu des pages, d’inclure des 
ressources multimédias dont des images, des formulaires de saisie, et des 
éléments programmables tels que des applets. Il permet de créer des documents 
interopérables avec des équipements très variés de manière conforme aux exigences 
de l’accessibilité du web.

    \subsection{Sources}

\begin{itemize}
 \item \url{http://fr.wikipedia.org/wiki/Json}
\end{itemize}

  \section{Conclusion}

JSON et XML sont les deux format qui semble les plus adaptés à nos besoin. 
JSON à un avantage au niveau de son interprétation par Javascript, une 
technologie certainement clé dans ce projet.

\chapter{Webographie}

  \section{Delicious}

Mon compte delicious : \url{http://www.delicious.com/binome.lyon/wge}


\end{document}          
